\section{Предсказание землетрясения}
Мы рассмотрели пару самых популярных прекурсоров и предложили свой. Кроме этого, для прекурсора $RTL$, задавая различные гиперпараметры, можно получить серию прекурсоров. Также был получен метод кластеризации землетрясений.

Опишем наш подход к предсказанию землетрясений:

\begin{enumerate}
\item Выделение подсистем: пространственных кластеров
\item Расчет прекурсоров в каждом кластере, регрессия прекурсоров $\setminus k$ кластеров на значения прекурсора в кластере $k$
\item Выделение наилучшего подмножества прекурсоров и решающего правила над ними
\end{enumerate}

Нам осталось описать алгоритм, который будет выделять наилучшее подмножество прекурсоров.

\subsection{Агрегация различных прекурсоров}
Значения различных прекурсоров представляют собой бинарные ответы, есть аномалия, по мнению прекурсора, или нет. Ответы можно агрегировать как логическое "и", логическое "или", либо как голосование. Опишем алгоритм выбора наилучшего подмножества:

Пусть $(X_{ft}) \in \{0; 1\}$ бинарные значение прекурсора $f$ из множества всех прекурсоров $f\in F$ и $(y_t)_{t=1}^T \in \{0; 1\}$ разметка, сильно ли произошло землетрясение.
\begin{itemize}
\item Разобьем временную ось на $K$ непересекающихся интервалов
\item Обозначим $K^{+}$ число успешных тревог, и за $K^{-}$ ложных
\item Для каждого $f\in F$ подсчитаем:
\begin{itemize}
\item Число успешных тревог в общем и уникальных $S^{+}_{f}, U^{+}_{f}$ соответственно
\item Число ложных тревог в общем и уникальных $S^{-}_{f}, U^{-}_{f}$ соответственно
\item Рассчитаем вероятности успешного и ложного срабатываний $p^{*} = \dfrac{S^{*}}{K^{*}}, * \in \{+;-\}$
\item Рассчитаем P-value $p_{f}^{+} = p_{f}^{-}$
\end{itemize}
\item Если гипотеза о равенстве отвергается, то такой предиктор $f$ остается.
\end{itemize}

\newpage

\section{Результаты}
\subsection{Япония}
\subsubsection*{Данные}
Землетрясения в окрестности Японии с 1990 по 2015 год. Целевыми землетрясениями считали землетрясения с магнитудой более $6$. Обучающая выборка содержала $500$ целевых событий и $185210$ землетрясений малой магнитуды.

\subsubsection*{Отобранные прекурсоры}
Перебор ввелся по множествам мощности 3, решающие правило: логическое или
\begin{itemize}
\item RTL, с параметрами окна $R_c = 200, T_c = 365, M_c = 3$ 
\item Предложенный прекурсор, в качестве ковариатов использовались средние магнитуды у 5 ближайщих соседей
\item Невязка регрессии RTL ближайщего кластера против RTL в остальных кластерах
\end{itemize}

Ядерные варианты предложенного прекурсора, к сожалению не вошли.

$\delta_c$ временной между предективным окном и целевым событием. Длина предиктивного окна: $300$ дней.

\subsubsection*{Результат}
\begin{table}[H]
\centering
\caption{Результат}
\label{my-label}
\begin{tabular}{ccccc}
$\delta_c$ & всего событий & \multicolumn{1}{l}{всего срабатываний} & предсказано успешно & ложные срабатывания \\
\textbf{15} & 80 & 1765 & 64 & 1701 \\
\textbf{20} & 80 & 1261 & 71 & 1190 \\
\textbf{30} & 80 & 985 & 70 & 915
\end{tabular}
\end{table}

\subsection{Италия}
\subsubsection*{Данные}
Землетрясения в окрестности Италии с 2000 по 2014 год. Целевыми землетрясениями считали землетрясения с магнитудой более $4.5$. Обучающая выборка содержала $100$ целевых событий и $2673$ землетрясений малой магнитуды.


\subsubsection*{Отобранные прекурсоры}
Перебор ввелся по множествам мощности 3, решающие правило: логическое или
\begin{itemize}
\item RTL, с параметрами окна $R_c = 150, T_c = 365, M_c = 1.$, т.е. фактически отсечки по минимальной магнитуде не было
\item  RTL, с параметрами окна $R_c = 150, T_c = 365, M_c = 3.$
\item Предложенный прекурсор,  в качестве ковариатов использовались средние магнитуды у 3 ближайщих соседей
\end{itemize}


$\delta_c$ временной между предективным окном и целевым событием. Длина предиктивного окна: $300$ дней.

\subsubsection*{Результат}
\begin{table}[H]
\centering
\caption{Результат}
\label{my-label}
\begin{tabular}{ccccc}
$\delta_c$ & всего событий & \multicolumn{1}{l}{всего срабатываний} & предсказано успешно & ложные срабатывания \\
\textbf{15} & 20 & 49  & 13 & 36 \\
\textbf{20} & 20 &  53 &  12 & 41 \\
\textbf{30} & 20 &  46 &  18 & 28
\end{tabular}
\end{table}

\subsection{Выводы}
\begin{itemize}
\item Не строго можно утверждать, что прекурсоры обладают предсказательной силой
\item Однако, на примере  Японии, хорошо видно, что нужны более частотные данные, чем каталог землетрясений. В Италии, где места землетрясений и их количество гораздо разряженее, относительно Японии, картина предсказаний значительно лучше.
\item RTL очень мощный прекурсор. Надо адаптировать его как ядро и попробовать использовать в ядровых методах
\end{itemize}