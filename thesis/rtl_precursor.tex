\section{RTL прекурсор}
\subsection{Обзор}
Энергия от деформации при движении плит распространяется на поверхность как волны. Отсюда можно предположить, что функция, отражающая "сейсмичность", будет непрерывна относительно координат. Введем такую функцию

$$
RTL(x,y,z,t)  = R(x,y,z,t)\cdot T(x,y,z,t)\cdot L(x,y,z,t)
$$

Каждая из компонент прекурсора улавливает изменения в пространственной и временной кластеризации землетрясения, а также в целом уровень сейсмической активности в окрестности точки $(x, y, z, t$.

\begin{equation*}
\text{Spatial-Temporal}\left\{\begin{aligned}
& R(x,y,z,t) = \left[\sum\limits_{i\in\mathcal{E}}\exp\left(-\dfrac{r_i}{r_0}\right)\right],~  r_i = \|(x,y,z)_{\text{current}} - (x,y,z)_{\text{earthquake i}} \|_2 \\
& T(x,y,z,t) = \left[\sum\limits_{i\in\mathcal{E}}\exp\left(-\dfrac{t-t_i}{t_0}\right)\right] \\
& L(x,y,z,t) = \left[\sum\limits_{i\in\mathcal{E}}\exp\left(\dfrac{l_i}{l_0}\right)\right] ,\\
&~\text{for Japan emperical relation} \log l_i = 0.5M_i - 1.8
\end{aligned}\right.
\end{equation*}


Окрестность $\mathcal{E}$ для землетрясения $(x,y,z,t)$ определяется как все землетрясения, попадающие в пространственно-временной цилиндр:

\begin{equation*}
\begin{aligned}
M_i > M_{\min} \\
r_i < R_{\max} = 2r_0 \\
0< t - t_i < T_{\max} = 2t_0
\end{aligned}
\end{equation*}

В литературе не описывается никаких способов автоматического подбора гиперпараметров, задающих окрестностей. По сути, задача такая же, как выбор оптимальной ширины ядра для восстановления плотности, и можно использовать $LOOCV$. Алгоритм оказывается очень чувствительным к гиперпараметрам.
\newpage
Следуя статье~\cite{rtl}, как прекурсоры также рассмотрим усредненное значение $RTL$

$$Q(x,y,z,t,\Delta t) = \dfrac{1}{m}\sum_{i=1}^{m}RTL(x,y,z,t_i),~t_i\in[t,\Delta t]$$ 

Кроме этого, я вычитал среднее и скользящее среднее. Это сделало прекурсор менее шумным.

\subsection{Определение критических значений}
У данного прекурсора нет вероятностного смысла, соответственно нельзя определить критические квантили. 
Поэтому возможно несколько стратегий по определению аномальных значений:
\begin{itemize}
\item One-class SVM обученный по фиксированному промежутку $[0;T]$
\item Эмпирические квантили гистограммы, которую можно перестраивать в скользящем окне
\item Критические квантили остатков модели $RTL_{t} = f(RTL_{t-1},\dots, (RTL_{t-k}) + \varepsilon_t$, в предположении их независимости и нормальности. 
\end{itemize}

Я использовал последний способ. В качестве модели $f()$ я использовал $SVM$ с rbf ядром и линейную регрессию, в обоих случаях использовалась $l_1$ регуляризация. 

