\section{B-value прекурсор}

\subsection{Обзор }
Прекурсор был введен в статье~\cite{gr_b_value} и известен также как Gutenberg–Richter law. 

$$
\log(N(>M)) = a - bM,~N(>M)\text{ общее число землетрясений с магнитудой больше $M$}
$$

Изначально был обнаружен чисто как эмпирический факт, позднее возникли различные обоснования в геофизике, обзор которых дан в статье~\cite{physic_b}.
Однако до сих пор нет подтверждения или обоснования, что пространственные и временные изменения в b-value действительно являются предвестниками сильных землетрясений. Некоторый эмпирический опыт можно обобщить как~\cite{b_one, b_two, b_three}:

\begin{itemize}
\item отклонение b-value от среднего уровня по пространству отвечает повышению уровня напряжения между тектоническими плитами
\item связь обратная, резкое уменьшение b-value отвечает сильному напряжению
\item отклонение b-value от среднего уровня по времени отвечает возможному сильному землетрясению
\item здесь связь прямая: сильным землетрясением предшествует положительное отклонение b-value от среднего уровня по времени 
\end{itemize}

Перечислим также недостатки, которые бы хотелось отметить:

\begin{itemize}
\item В статьях, где демонстрируется пары "аномалия в b-value $\to$ сильное землетрясение" сроки между этими событиями варьируются от 3 месяцев до нескольких лет. Однако, модель не способна это никак учесть или дать оценку сроку.
\item Прекурсор основывается на пуассоновском распределении числа землетрясений. Однако в силу существования форшоков и афтершоков (заметим, что определения этих терминов, основанного на физике процесса -- нет) реальный процесс не представляет собой процесс переключения параметра интенсивности между постоянными значениями. Часто встречающийся подход к решению этой проблемы - разделение каталога на независимые землетрясения и афтершоки. После разделения последние исключаются из рассмотрения.  В то же время есть работы, показывающие, что такое редактирование каталога создает искусственные аномалии в данных. 
\end{itemize}

\subsection{Распределение статистики}
При использовании $b-value$ все статьи ссылаются на оригинальную статью~\cite{b_val_distrib}, которая уже недоступна. Так как мы будем использовать эту статистику, приведем вывод ее распределения.

Покажем вспомогательное утверждение. Пусть $f(x;\theta)$ семейство распределений, c носителем $x$. Рассмотрим $s(x;\theta) = \frac{\partial}{\partial \theta}\log f(x;\theta)$.

\begin{equation*}
\begin{aligned}
& 1 = \int f(x;\theta) dx \\
& 0 = \dfrac{\partial}{\partial \theta}\int f(x;\theta) dx = \int \dfrac{\partial}{\partial \theta}  f(x;\theta) dx = \\
& = \int \dfrac{\partial}{\partial \theta} \log f(x;\theta)f(x;theta) dx = \mathbb{E}_{\theta} s(x;\theta)
\end{aligned}
\end{equation*}

Тогда для экспоненциального семейства распределений:

\begin{equation*}
\begin{aligned}
& \log f(x;\theta) = l_{\theta} = \langle \theta, T(x) \rangle - c(\theta)  \\
& \dfrac{\partial}{\partial \theta} l_{\theta} = \langle \textbf{1}, T(x) \rangle - \dfrac{\partial }{\partial \theta}c(\theta) \\
& \Exp \left(\langle \textbf{1}, T(x) \rangle - \dfrac{\partial }{\partial \theta}c(\theta)\right) = 0 \\
& \Exp T(x)_{j} = \dfrac{\partial}{\partial \theta_{j}}c(\theta)
\end{aligned}
\end{equation*}

Экспоненциальное распределение входит в экспоненциальное семейство распределений. Тогда через дифференцирование константы найдем $MLE$ оценку и информацию Фишера. Таже воспользуемся асимптотически нормальным распределеним $MLE$ оценки.

\begin{equation*}
\begin{aligned}
& f(M, b) = b\exp\left(-b(M-m_0)\right) \\
& \hat{b} = \dfrac{1}{\overline{M} - M_0} \\
& \sqrt{n}(\hat{b} - b) \sim \mathcal{N}(0, b^2)
\end{aligned}
\end{equation*}
