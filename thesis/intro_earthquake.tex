\section{Предсказание землетрясений}

\subsection{Постановка задачи}
Пусть $t$ индексирует время. Мы наблюдаем временной ряд $Y_t$, где $Y_i = [\text{целевой событие в момент} t=i]$. Также мы наблюдаем признаки $X_ {t;k}$, для каждого фиксированного $k=k'$, $X_{t, k'}$ многомерный временной ряд. Неформально наша цель состоит в заблаговременном предсказании целевого события $Y$ по признаковому описанию $X$. Иначе говоря: по наблюдаемой истории $(X_{\{t, x\}}; Y_t)$ до некоторого момента $T ,~t\in [0;T]$, необходимо поднять тревогу в окне $[T; T+\delta_1]$ при том, что целевое событие происходит в окне  $[T+\delta_1; T+\delta_1+\delta_2]$. Примерами таких задач может служить:

\begin{itemize}
\item Предсказание поломок в сложных технических системах. Тогда $k$ индексирует датчики в такой системе и $X_{t;k}$ представляет из себя показания датчика.
\item Предсказание природных катастроф. Тогда $k$ может представлять или координаты $(lat;long)$ либо номер кластера.
\end{itemize}

В данной работе мы фокусируемся на предсказании землетрясений. Необходимо определить, что тогда является целевым событием. Существует несколько вариантов:

\begin{itemize}
\item Предсказание начала серии афтершоков, т.е. последовательности землетрясений после самого сильного (main shock). С такой задачей хорошо справляются различные ETAS модели.
\item Предсказание сильных землетрясений на долгосрочном горизонте (годы) (модели seismic gap)
\item Предсказание сильных землетрясений на среднесрочном горизонте (месяцы)
\end{itemize}

Мы займемся предсказанием сильных землетрясений на среднесрочном горизонте. Под сильными имеются в виду землетрясения с магнитудой выше $M_c = 5$.
Перечислим сложности, с которыми связаны предсказания землетрясений:
\begin{itemize}
\item Очень не сбалансированная выборка \\

С 1990-2016 в Японии, всего 247204 землетрясений, распределение по магнитудам:
\begin{table}[H]
\centering
\begin{tabular}{cc}
\hline
\multicolumn{1}{|c|}{\textbf{Магнитуда}} & \multicolumn{1}{c|}{\textbf{Число землетрясений с большей магнитдой}} \\ \hline
5.5 & 759 (0.3\%) \\
6.5 & 87 (0.035\%) \\
7.5 & 7 (0.002\%) \\
8 & 1
\end{tabular}
\end{table}

\item Искусственные аномалии в каталоге из-за изменения сети сейсмостанций
\item Чем выше магнитуда ожидаемого распределения, тем шире окно между прекурсором и им (более года для землетрясения магнитудой больше 7)
\item Quasi-Periodic, Poisson, Clustered? Есть свидетельства как против, так и за по каждой гипотезе
\item Сложно валидировать методики, т.к. прогнозы могут быть отложены во времени~\cite{work}: "there is a 0.62 probability of a major, damaging $[M \geq 6.7]$ earthquake striking the greater San Francisco Bay Region over the next 30 years (2002-2031)"
\end{itemize}

\subsection{Прекурсоры}
Базовая предпосылка: появление сильных землетрясений робастно описывается функционалами  $F_k(t)$, каждый из которых описывает какой-либо статистический паттерн сейсмического поля и рассчитывается в скользящем окне $(t - s, t)$. Если аномальные значения такого функционала отвечают наступлению сильного землетрясения, то будем называть такой функционал прекурсором.

Прекусоры строятся как статистики, зависящие от:
\begin{itemize}
\item изменения сейсмической активности
\item изменение кластеризации землетрясений во времени или пространстве 
\item изменения зависимости частоты землетрясений от магнитуды
\end{itemize}

Возможные данные, на которых строят прекурсоры:

\begin{itemize}
\item Геомагнитное поле
\item Концентрация радона
\item Изменений уровня/температуры подземных вод
\item Деформация земной поверхности
\item Изменение в бассейнах нефти
\item Микросейсмичность (выскочастотные ряды)
\item \textbf{Мы работаем с каталогом землетрясений}: (latitude, longitude, depth, magnitude, Year-Month-Day-Hour-Seconds), относительно низкочастотные
\end{itemize}

Далее рассмотрим популярные прекурсоры и предложим свой.